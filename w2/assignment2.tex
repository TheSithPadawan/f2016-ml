
\documentclass[12pt]{article}
\usepackage{enumitem}
\usepackage{mathtools}
\usepackage{amsthm}
\usepackage{graphicx}
\graphicspath{ {images/} }
\begin{document}

\title{Assignment 2}
\author{Darwin Ding}
\maketitle

\section*{Exercise 1.8}
The probability of picking a marble out of the bag and getting red is 0.9, and we are looking for the probability that we sample 10 random marbles and get 1 or 0 red marbles.

The probability of getting 0 red marbles is $(.1)^{10} = 10^{-10}$, and the probability of getting 1 red marble is $(.1)^9 * .9 * 10$, because we pulled 9 green marbles (with probability .1) and 1 red marble (probability .9), and there were 10 ways to do that.

The probability of doing either of those things is the sum of those two probabilities is $\boldsymbol{9.1 * 10^{-9}}$

\section*{Exercise 1.9}
Using Hoeffding's Inequality, $\mu = .9$. Since we're looking for $v \le .1$, $\epsilon \ge .9 - .1 = .8$. By definition, also, $N = 10$.

\begin{gather*}
P[|v - \mu| \ge \epsilon] \le 2e^{-2\epsilon^{2}N}
\\ \le 2e^{-2 * .8^2 * 10}
\\ \le \boldsymbol{5.52 * 10^{-6}}
\end{gather*}

\section*{Exercise 1.10}
\begin{enumerate}[label=(\alph*)]
	\item After running the experiment a single time, $\mu$ for $v_0$ = \textbf{0.3}, $\mu$ for $v_{rand}$ = \textbf{0.5} and $\mu$ for $v_{min}$ = \textbf{0}
\end{enumerate}

\end{document}